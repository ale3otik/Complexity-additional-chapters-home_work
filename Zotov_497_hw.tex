\documentclass[a4paper]{article}
\usepackage[margin=1.0in]{geometry}

\usepackage[l2tabu,orthodox]{nag}
\usepackage{indentfirst}
\usepackage{amssymb,amsfonts}
\usepackage[]{mathtools}
\usepackage{cmap}
\usepackage[T2A]{fontenc}
\usepackage[utf8]{inputenc}
\usepackage{ucs}
\usepackage[russian,english]{babel}
\usepackage[babel = true]{microtype}
\usepackage{graphicx}
\usepackage[colorinlistoftodos, textsize=tiny]{todonotes}

\usepackage{color}
\definecolor{darkblue}{rgb}{0,0,.75}
\definecolor{darkred}{rgb}{.7,0,0}
\definecolor{darkgreen}{rgb}{0,.7,0}

\usepackage[
    draft = false,
    unicode = true,
    colorlinks = true,
    allcolors = blue,
    hyperfootnotes = true
]{hyperref}
\usepackage{amsmath}
\usepackage{amsthm}
    \theoremstyle{plain}
    \newtheorem{theorem}{Теорема}
    \newtheorem{task}{Задача}
    \newtheorem{lemma}{Лемма}
    \newtheorem{proposition}{Утверждение}
    \newtheorem{corollary}{Следствие}
    \theoremstyle{definition}
    \newtheorem{definition}{Определение}
    \newtheorem*{notation}{Обозначение}
    \newtheorem{example}{Пример}
        \newtheorem*{answer}{Ответ}
    \newtheorem*{draft}{Черновик ответа}

\title{Зачетные задачи}

\author{Зотов Алексей 497}

\date{\today}

\begin{document}
\maketitle

\begin{task}
Построить систему интерактивных доказательств для языка \textsf{GI-NO-EQUAL-CLASSES} = $\{(G_1, \ldots G_m) \ |$  в разбиении этого набора графов на классы эквивалентности по отношению изоморфизма нет двух классов одинакового размера$\}$
\end{task}
\begin{answer}  
    Мы уже знаем, что $\textsf{GNI} \in \mathbf{IP}$ и будем это использовать. Также $\textsf{GI} \in \mathbf{NP}$. $M = \{1,\ldots,m\}$
    \item Рассмотрим такой протокол :
    \begin{enumerate}
        \item $\forall i \in 1,\ldots , m$  верификатор $V$ посылает пруверу $P$ индекс $i$ соответствующий $G_i$.
        \item $P$ возвращает $X_i = \{(k,S_{ki}) | G_k \cong G_i\}$ - множество индексов графов, изоморфных $G_i$ и соответствующие сертификаты изоморфности. $X_i = (K_i,S_i)$ - обозначение.
        \item $V$ проверяет полученные сертификаты. 
        \item $\forall j : j \notin K_i$ верификатор $V$ инициирует протокол проверки, что $G_i \ncong G_j$, причем вероятность ошибки $p_{ij} \leq \frac{1}{3m^2}$. 
        \item Повторяется с пункта $(1)$, пропуская те индексы, для которых уже найден класс изоморфности.
        \item $V$ проверяет, что все классы получились разного размера.
    \end{enumerate}
    Докажем, что алгоритм корректен: 
    \begin{itemize}
        \item если $(G_1, \ldots G_m) \in \textsf{GI-NO-EQUAL-CLASSES}$, тогда каждый на каждой итерации прувер будет действовать наилучшим образом, положительная проверка на изоморфность и неизоморфность проходит без ошибок (с вероятностью 1). 
        \item если $(G_1, \ldots G_m) \notin \textsf{GI-NO-EQUAL-CLASSES}$, тогда $P$ не может неизоморфные графы отнести в один класс, но может попробовать изоморфные графы разбить по разным классам, воспользовавшись наличем ошибки при проверке $G_i \ncong G_j$. Таких проверок не больше $m^2$, значит $P_{err} \leq \sum p_{ij} \leq m^2 \cdot \frac{1}{3m^2} = \frac{1}{3}$.
    \end{itemize}
\end{answer}


\end{document}
