\documentclass[a4paper]{article}
\usepackage[margin=1.0in]{geometry}

\usepackage[l2tabu,orthodox]{nag}
\usepackage{indentfirst}
\usepackage{amssymb,amsfonts}
\usepackage[]{mathtools}
\usepackage{cmap}
\usepackage[T2A]{fontenc}
\usepackage[utf8]{inputenc}
\usepackage{ucs}
\usepackage[russian,english]{babel}
\usepackage[babel = true]{microtype}
\usepackage{graphicx}
\usepackage[colorinlistoftodos, textsize=tiny]{todonotes}

\usepackage{color}
\definecolor{darkblue}{rgb}{0,0,.75}
\definecolor{darkred}{rgb}{.7,0,0}
\definecolor{darkgreen}{rgb}{0,.7,0}

\usepackage[
    draft = false,
    unicode = true,
    colorlinks = true,
    allcolors = blue,
    hyperfootnotes = true
]{hyperref}
\usepackage{amsmath}
\usepackage{amsthm}
    \theoremstyle{plain}
    \newtheorem{theorem}{Теорема}
    \newtheorem*{task}{Задача}
    \newtheorem{lemma}{Лемма}
    \newtheorem{proposition}{Утверждение}
    \newtheorem{corollary}{Следствие}
    \theoremstyle{definition}
    \newtheorem{definition}{Определение}
    \newtheorem*{notation}{Обозначение}
    \newtheorem{example}{Пример}
        \newtheorem*{answer}{Ответ}
    \newtheorem*{draft}{Черновик ответа}

\title{Зачетные задачи}

\author{Зотов Алексей 497}

\date{\today}

\begin{document}
\maketitle

%%%%%%%%%%%%%%%%%%%%%%%%%%%%%%%%%%%%%%%%%%%%%% TASK  1 %%%%%%%%%%%%%%%%%%%%%%%%%%%%%%%%%%%%%%%%%%%%%%
\begin{task}{\textbf 1.}
Построить систему интерактивных доказательств для языка \textsf{GI-NO-EQUAL-CLASSES} = $\{(G_1, \ldots G_m) \ |$  в разбиении этого набора графов на классы эквивалентности по отношению изоморфизма нет двух классов одинакового размера$\}$
\end{task}
\begin{answer}  
    Мы уже знаем, что $\textsf{GNI} \in \mathbf{IP}$ и будем это использовать. Также $\textsf{GI} \in \mathbf{NP}$. $M = \{1,\ldots,m\}$
    \item Рассмотрим такой протокол :
    \begin{enumerate}
        \item $\forall i \in 1,\ldots , m$  верификатор $V$ посылает пруверу $P$ индекс $i$ соответствующий $G_i$.
        \item $P$ возвращает $X_i = \{(k,S_{ki}) | G_k \cong G_i\}$ - множество индексов графов, изоморфных $G_i$ и соответствующие сертификаты изоморфности. $X_i = (K_i,S_i)$ - обозначение.
        \item $V$ проверяет полученные сертификаты. 
        \item $\forall j : j \notin K_i$ верификатор $V$ инициирует протокол проверки, что $G_i \ncong G_j$, причем вероятность ошибки $p_{ij} \leq \frac{1}{3m^2}$. 
        \item Повторяется с пункта $(1)$, пропуская те индексы, для которых уже найден класс изоморфности.
        \item $V$ проверяет, что все классы получились разного размера.
    \end{enumerate}
    Докажем, что алгоритм корректен: 
    \begin{itemize}
        \item если $(G_1, \ldots G_m) \in \textsf{GI-NO-EQUAL-CLASSES}$, тогда каждый на каждой итерации прувер будет действовать наилучшим образом, положительная проверка на изоморфность и неизоморфность проходит без ошибок (с вероятностью 1). 
        \item если $(G_1, \ldots G_m) \notin \textsf{GI-NO-EQUAL-CLASSES}$, тогда $P$ не может неизоморфные графы отнести в один класс, но может попробовать изоморфные графы разбить по разным классам, воспользовавшись наличем ошибки при проверке $G_i \ncong G_j$. Таких проверок не больше $m^2$, значит $P_{err} \leq \sum p_{ij} \leq m^2 \cdot \frac{1}{3m^2} = \frac{1}{3}$.
    \end{itemize}
\end{answer}


%%%%%%%%%%%%%%%%%%%%%%%%%%%%%%%%%%%%%%%%%%%%%% TASK  5 %%%%%%%%%%%%%%%%%%%%%%%%%%%%%%%%%%%%%%%%%%%%%%
\begin{task}{\textbf 5.}
Постройте систему интерактивных доказательств с общими случайными битами для языка \textsf{GROUP-NI} $=\{G_0, G_1 \ |\ G_0,\ G_1$ --- табилцы умножения двух неизоморфных конечных групп$\}$
\end{task}
\begin{answer}  
Проверить, что данные таблицы это таблицы умножения групп, верификатор может без прувера за $O(n^2)$. Достаточно проверить \textit{ассоциативность, наличие единицы и обратимость всех элементов}. Нужно проверить их неизоморфность. 
\\Рассмотрим $S = \{(H,\sigma) | H \cong G_i , i \in \{0,1\}, \sigma \in \text{Aut} H\}$. Тогда, если $G_0 \cong G_1$, то $|S| = n!$, иначе $|S| = 2 \cdot n!$. Воспользуемся семейством попарно независимых полиномиально вычислимых хеш-функций $H_{n,k}: 2^\mathbf{N} \to 2^\mathbf{K}$.  А дальше как на лекции! TODO...
\end{answer}

%%%%%%%%%%%%%%%%%%%%%%%%%%%%%%%%%%%%%%%%%%%%%% TASK  6 %%%%%%%%%%%%%%%%%%%%%%%%%%%%%%%%%%%%%%%%%%%%%%
\begin{task} {\textbf 6.}
Пусть $G$ является генератором псевдослучайных чисел. Рассмотрим следующие модификации:
\begin{itemize}
\item $G'(s) = \left\{
     \begin{array}{lr}
       0^{|G(s)|},&  \text{если } $s$ \text{ содержит ровно } \frac{|s|}{2} \text{ единиц}\\
       G(s) ,& \text{иначе}
     \end{array}
   \right.$
\item $G'(s) = \left\{
     \begin{array}{lr}
       0^{|G(s)|},&  \text{если } $s$ \text{ содержит ровно } \frac{|s|}{3} \text{ единиц}\\
       G(s) ,& \text{иначе}
     \end{array}
   \right.$ 
\end{itemize}
Какие из этих функций являются генераторами псведослучайных чисел и почему?
\end{task}

\begin{answer}
Считаем $n = |s|$. В обоих случая полиномиальная вычислимость $G'(s)$ очевидна. Нужно проверить пункт (2) определения.
\begin{enumerate}
\item \begin{equation} G'(s) = \left\{
     \begin{array}{lr}
       0^{|G(s)|},&  \text{если } $s$ \text{ содержит ровно } \frac{|s|}{2} \text{ единиц}\\
       G(s) ,& \text{иначе}
     \end{array}
   \right. 
   \end{equation}
   $G'(s)$ - не является ГПСЧ. \\
   В $s$ ровно $\frac{|s|}{2}$ единиц в $C_{n}^\frac{n}{2}$ различных $s$. Считая, что $s \sim U_n$ и воспользовавшись тем, что для достаточно больших $n$ выполнено $C_{n}^\frac{n}{2} > \frac{2^n}{n+1}$, получим:
   \begin{equation}
       P(G(s) = 0^{p(n)}) \geq \frac{C_{n}^\frac{n}{2}}{2^n} \geq \frac{1}{n+1}  \quad n \geq N_0
   \end{equation}
   Воспользуемся определeнием вычислительной неотличимости, $y_n \sim U_{p(n)}$, пусть $\{D_n\}$ - такое симейство схем, что $D_n(x) = 1 \iff x = 0^n$. Получим : \\
   $|P\{D_n(G'(s)) = 1\} - P\{D_n(y_n)) = 1\}| \geq \frac{1}{n+1} - \frac{1}{2^n} \geq \frac{1}{2(n+1)}.$ при $n \geq 10$. 
   Также $\frac{1}{2(n+1)} \geq  \frac{1}{2(p(n)+1)}$ при $n > N_{p}$.
   То есть мы получили, что $\exists \{D_n\}$ , $\exists q(p(n)) = \frac{1}{2(p(n)+1)}$  $\forall N \exists n > N : |P\{D_n(G'(s)) = 1\} - P\{D_n(y_n)) = 1\}| \geq \frac{1}{q(p(n))}$. Значит $y_n$ и $G'(s)$ - не являются вычислительно неотличимыми. Значит $G'(s)$ - не является ГПСЧ.

\item $G'(s) = \left\{
     \begin{array}{lr}
       0^{|G(s)|},&  \text{если } $s$ \text{ содержит ровно } \frac{|s|}{3} \text{ единиц}\\
       G(s) ,& \text{иначе}
     \end{array}
   \right.$
$G'(s)$ - не является ГПСЧ. \\ 
В $s$ ровно $\frac{|s|}{3}$ единиц в $C_{n}^\frac{n}{3}$ различных $s$. Воспользуемся формулой Стирлинга: 
\begin{equation}
    C_{n}^{\frac{n}{3}} = \frac{n!}{\frac{n}{3}! \frac{2n}{3}!} \sim \frac{3}{\sqrt{4 \pi n}} \frac{3^n}{2^{\frac{2n}{3}}}
\end{equation}
Обозначим событие $X = \{$ в $s$ ровно $\frac{|s|}{3}$ единиц$\}$. Тогда, считая $s \sim U_n$, получим : 
\begin{equation}
    P\{G'(s) \neq G(s) \} \leq P\{X\} \sim \frac{3^n}{2^{\frac{5n}{3}}}
\end{equation}
$\frac{3^n}{2^{\frac{5n}{3}}} = e^{n(\ln 3 - \frac{5}{3} \ln 2)}$.  Заметим, что $\ln 3 - \frac{5}{3} \ln 2 = c < 0$.
Т.е. $P\{G'(s) \neq G(s) \} \sim \frac{3}{2\sqrt{\pi n}} e^{cn}$. 
Значит $\exists N \forall n > N:$ $P\{G'(s) \neq G(s) \} \leq \frac{3}{\sqrt{\pi n}} e^{c_0 n} \leq e^{cn},\quad c_0, c < 0$. \\

Так как $G(s)$ - ГПСЧ, то $y_n \sim U_{p(n)}, \forall \{D_n\} \forall q_1(x) \text{ - полином } \exists N \forall n > N : $\\ 
$|P\{D_n(G(s)) = 1 \} - P\{D_n(y_n) = 1\}| < \frac{1}{q_1(p(n))}$. 

Воспользуемся определeнием вычислительной неотличимости : \\
$y_n \sim U_{p(n)}, \forall \{D_n\} \forall q(x) \text{ - полином } \exists q_1(x) = \frac{q(x)}{2},  \exists N \forall n > N : 
     |P\{D_n(G'(s)) = 1 \} - P\{D_n(y_n) = 1\}| 
     \leq |P\{D_n(G'(s)) = 1 \} - P\{D_n(G(s)) = 1 \}| + |P\{D_n(G(s)) = 1 \} - P\{D_n(y_n) = 1\}|
     < e^{cn} + \frac{1}{q_1(p(n))} < \frac{1}{q(p(n))}$ 

Получили, что $G'(s)$ и $y_n$ вычислительно неотличимы. Значит $G'(s)$ - ГПСЧ.
\end{enumerate}
\end{answer}





%%%%%%%%%%%%%%%%%%%%%%%%%%%%%%%%%%%%%%%%%%%%%TASK 7%%%%%%%%%%%%%%%%%%%%%%%%%%%%%%%%%%%%%%%%%%%%%$
\begin{task}{\textbf 7.}
Обобщённым судоку называется такая задача: в квадрате $n^2 \times n^2$ в некоторых клетках расставлены числа от $1$ до $n^2$. Вопрос: можно ли заполнить оставшиеся клетки числами от $1$ до $n^2$, так чтобы в каждой строке, в каждом столбце, а также в каждом из $n^2$ "выровненных" квадратов $n \times n$ каждое число встречалось по одному разу. В стандартном судоку $n=3$. Известно, что эта задача $\mathbf{NP}$-полна. Предложите протокол доказательства существования решения с вычислительно нулевым разглашением, не использующий сводимость к какой-либо другой задаче.
\end{task}

\begin{draft}
Идея: Исходная таблица $T_0$. $P$ выбирает случайную перестановку $\sigma \in S_{n^2}$, записывает решение в таблицу $T_1$, применяет $\sigma$ к числам $\{1,\ldots n^2\}$ из таблицы $T_1$. "Закрывает" таблицу и перестановку $\sigma$. $V$ использует сколько нужжно случайных бит, выбирает строку, столбец или квадрат, и просит прувера открыть в $T_1$. Также выбирает случайную позицию $(i,j)$ в исходной таблице такую, что $T_0[i,j] = x$ (т.е. в $T_0[i,j]$ записано некоторое известное число $x$) , и просит открыть $\sigma (x)$. Проверяет на корректность строку, столбец или квадрат соответственно, а также проверяет, что $T_1[i,j] = \sigma(x)$. Случайные биты, открытие-закрытие как на лекции. Вероятность найти ошибку за 1 шаг $p \geq \frac{1}/n^10$.  
\end{draft}


\end{document}
