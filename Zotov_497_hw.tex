\documentclass[a4paper]{article}
\usepackage[margin=1.0in]{geometry}

\usepackage[l2tabu,orthodox]{nag}
\usepackage{indentfirst}
\usepackage{amssymb,amsfonts}
\usepackage[]{mathtools}
\usepackage{cmap}
\usepackage[T2A]{fontenc}
\usepackage[utf8]{inputenc}
\usepackage{ucs}
\usepackage[russian,english]{babel}
\usepackage[babel = true]{microtype}
\usepackage{graphicx}
\usepackage[colorinlistoftodos, textsize=tiny]{todonotes}

\usepackage{color}
\definecolor{darkblue}{rgb}{0,0,.75}
\definecolor{darkred}{rgb}{.7,0,0}
\definecolor{darkgreen}{rgb}{0,.7,0}

\usepackage[
    draft = false,
    unicode = true,
    colorlinks = true,
    allcolors = blue,
    hyperfootnotes = true
]{hyperref}
\usepackage{amsmath}
\usepackage{amsthm}
    \theoremstyle{plain}
    \newtheorem{theorem}{Теорема}
    \newtheorem*{task}{Задача}
    \newtheorem{lemma}{Лемма}
    \newtheorem{proposition}{Утверждение}
    \newtheorem{corollary}{Следствие}
    \theoremstyle{definition}
    \newtheorem{definition}{Определение}
    \newtheorem*{notation}{Обозначение}
    \newtheorem{example}{Пример}
        \newtheorem*{answer}{Ответ}
    \newtheorem*{draft}{Черновик ответа}

\title{Зачетные задачи}

\author{Зотов Алексей 497}

\date{\today}

\begin{document}
\maketitle

%%%%%%%%%%%%%%%%%%%%%%%%%%%%%%%%%%%%%%%%%%%%%% TASK  1 %%%%%%%%%%%%%%%%%%%%%%%%%%%%%%%%%%%%%%%%%%%%%%
\begin{task}{\textbf 1.}
Построить систему интерактивных доказательств для языка \textsf{GI-NO-EQUAL-CLASSES} = $\{(G_1, \ldots G_m) \ |$  в разбиении этого набора графов на классы эквивалентности по отношению изоморфизма нет двух классов одинакового размера$\}$
\end{task}
\begin{answer}  
    Мы уже знаем, что $\textsf{GNI} \in \mathbf{IP}$ и будем это использовать. Также $\textsf{GI} \in \mathbf{NP}$. $M = \{1,\ldots,m\}$
    \item Рассмотрим такой протокол :
    \begin{enumerate}
        \item $\forall i \in 1,\ldots , m$  верификатор $V$ посылает пруверу $P$ индекс $i$ соответствующий $G_i$.
        \item $P$ возвращает $X_i = \{(k,S_{ki}) | G_k \cong G_i\}$ - множество индексов графов, изоморфных $G_i$ и соответствующие сертификаты изоморфности. $X_i = (K_i,S_i)$ - обозначение.
        \item $V$ проверяет полученные сертификаты. 
        \item $\forall j : j \notin K_i$ верификатор $V$ инициирует протокол проверки $(G_i,G_j) \in \textbf{GNI}$), c вероятностью ошибки $p_{ij} \leq \frac{1}{3}$. 
        \item Алгорим повторяется с пункта $(1)$, игнорируя те индексы, для которых уже найден класс изоморфности.
        \item $V$ проверяет, что все классы получились разного размера. Возвращает \textsf{True}, если на каждый шаг протокола корректный (проверка сертификатов, проверка на $G_i \ncong G_j$), и полученные классы разного размера. Иначе \textsf{False}.
    \end{enumerate}
    Докажем, что алгоритм корректен: 
    \begin{itemize}
        \item если $X = (G_1, \ldots G_m) \in \textsf{GI-NO-EQUAL-CLASSES}$, тогда каждый на каждой итерации прувер будет действовать корректно, положительная проверка на изоморфность и неизоморфность проходит без ошибок (с вероятностью 1). 
        \item если $X = (G_1, \ldots G_m) \notin \textsf{GI-NO-EQUAL-CLASSES}$, тогда $P$ не может неизоморфные графы отнести в один класс(т.к. проверка сертификатов детерминированная), но может попробовать изоморфные графы разбить по разным классам, воспользовавшись наличем ошибки при проверке $G_i \ncong G_j$. На каждой такой проверке вероятность обмануть верификатор $p_{ij} \leq \frac{1}{3}$. Значит вероятность ошибочно принять $X$: $P_{\text{err}} \leq \frac{1}{3}$. 
    \end{itemize}
\end{answer}



%%%%%%%%%%%%%%%%%%%%%%%%%%%%%%%%%%%%%%%%%%%%%% TASK  4 %%%%%%%%%%%%%%%%%%%%%%%%%%%%%%%%%%%%%%%%%%%%%%
\begin{task}{\textbf 4.}
Постройте систему интерактивных доказательств с общими случайными битами для языка \textsf{GROUP-NI} $=\{G_0, G_1 \ |\ G_0,\ G_1$ --- табилцы умножения двух неизоморфных конечных групп$\}$
\end{task}
\begin{answer}  
Проверить, что данные таблицы это таблицы умножения групп, верификатор может без прувера за $O(n^2)$. Достаточно проверить \textit{ассоциативность, наличие единицы и обратимость всех элементов}. Нужно проверить их неизоморфность. 
\\Рассмотрим $S = \{(H,\sigma) | H \cong G_i , i \in \{0,1\}, \sigma \in \text{Aut} H\}$. Тогда, если $G_0 \cong G_1$, то $|S| = n!$, иначе $|S| = 2 \cdot n!$. Каждую группу порядка $n$ можно записать двоичным числом длины $m$, где $m = p(n)$. Обозначим $K = 2 n!$, имеем $S \subset \{0,1\}^m$. Выберем $k$ таким, что $ 2^{k-2} \leq K \leq 2^{k-1}$. Рассмотрим такой протокол: \\
\begin{itemize} 
\item $V$ выбирает случайную хэш-функцию $h$ из семейства попарно независимых полиномиально вычислимых(от $m,k$) хеш-функций
 $H_{m,k}: 2^\mathbf{m} \to 2^\mathbf{k}$. Также выбирает случайный $y \in {0,1}^k$. Отправляет пруверу $P$  пару $(h,y)$ (значит можно считать,что случайные биты - общие).
\item $P$ выбирает $x \in S : h(x) = y$. Возвращает верификатору пару $(x,s)$, где $s$ - сертификат $x \in S$.
\item $V$ проверяет сертификат $s$ и $h(x) = y$. Принимает доказательство если проверки корректные. 
\end{itemize}
Покажем корректность протокола. Пусть $p = \frac{|S|}{2^k}$. 
\begin{itemize} 
\item  $P_{h,y} \{ \exists x \in S : h(x) = y \} \leq p$ - так как $|h(S)| \leq |S|$.
\item Рассмотрим $E_x$ - событие $\{ h(x) = y\}$. $Pr \{ \cup_{x \in S} E_x \} \geq \sum_{x \in S} Pr\{E_x\} 
- \sum_{x < x' \in S} Pr \{ E_x \cap E_{x'} \} = \frac{|S|}{2^k} - \frac{|S|(|S| - 1)}{2} \frac{1}{2^{2k}} > p(1 - \frac{p}{2}) \geq 
 \geq \frac{3}{4} p$ (так как $p \leq \frac{1}{2}$ из-за выбора $p$ и ограничения сверху на размер $S$).   
\end{itemize}
Получили $\frac{3}{4} p \leq  P_{h,y} \{ \exists x \in S : h(x) = y \} \leq p$, значит :
\begin{itemize} 
\item Если $|S| \geq K$, то $\frac{3}{4} p_0 \leq  P_{h,y} \{ \exists x \in S : h(x) = y \}$
\item Если $|S| \leq \frac{K}{2}$, то $P_{h,y} \{ \exists x \in S : h(x) = y \} \leq p \leq \frac{p_0}{2} < \frac{3}{4} p_0$
 
Для разных случаев получили некоторый вероятностый зазор, который можно увеличить полиномильным числом повторений протокола. 

\end{itemize}
\end{answer}


%%%%%%%%%%%%%%%%%%%%%%%%%%%%%%%%%%%%%%%%%%%%%% TASK  2  %%%%%%%%%%%%%%%%%%%%%%%%%%%%%%%%%%%%%%%%%%%%%%
\begin{task} \textbf{2.}
Пусть есть $m=n(n-1)/2$ булевых схем полиномиального размера $\phi_1, \ldots, \phi_m$ со входом длины $k$ и одним выходом. Для каждого $x \in \{0,1\}^k$ рассмотрим граф $G_x$ на $n$ вершинах, матрица смежности которого задана результатами работы схем $\phi_j$ на входе $x$. Рассмотрим множество графов $\mathcal{G}=\{G_x | x \in \{0,1\}^k\}$. Пусть мы хотим отделить наборы $\phi_1, \ldots, \phi_m$, когда в множестве $\mathcal{G}$ менее $C$ различных попарно неизоморфных графов от наборов, когда их хотя бы $D$ ($D \ge C$). Существует ли интерактивная система доказательств, которая делает это при $C=D$? Можете ли вы её построить? Если не можете, то попробуйте её построить для случая $C=D/2$.
\end{task}

\begin{answer}
Да, такой протокол существует. $\mathbf{L} = \{\{(\phi_1 \ldots \phi_m)\}: $ в $\mathcal{G}$ содержится менее $D$ попарно неизоморфных графов$\}$. $\mathbf{L} \in \mathbf{PSAPCE}$. Можно посчитать количество классов изоморфности графов на полиномиальной памяти, для этого храним текущее число классов, для каждого не рассмотренного графа $G_i$, перебираем все уже рассмотренные $G_j$ (перебор графов, или наборов булевых формул это одно и то же), если $\forall j \leq i : G_j \ncong G_i$, то увеличиваем число классов эквивалентности. Полученное число классов эквивалентности и будет ответом. Дальше сравним его с $D$. Значит $\mathbf{L} \in \mathbf{PSAPCE}$. Знаем $\mathbf{IP} = \mathbf{PSAPCE}$. Значит нужный протокол из $\mathbf{IP}$ существует. 

Для $C = D/2$ подойдет протокол подробно описанный выше, в задаче (4).
\end{answer}



%%%%%%%%%%%%%%%%%%%%%%%%%%%%%%%%%%%%%%%%%%%%%% TASK  3  %%%%%%%%%%%%%%%%%%%%%%%%%%%%%%%%%%%%%%%%%%%%%%
\begin{task}{\textbf 3.}
Пусть $S \in \mathbf{NP}$. Обозначим через $S_n$ множество $S \cap \{0,1\}^n$. Постройте систему интерактивных доказательств, получающую на вход число $K$, такую что если $|S_n| > K$, то прувер убеждает верификатора с вероятностью $1$ (а не $2/3$, как на лекции), а если $|S_n|<K/2$, то прувер убеждает с вероятностью не больше $1/3$. Можно ли заменить $K/2$ на $0.99K$? (Аргумент полинома во времени работы верификатора - это $n$).
\end{task} 

\begin{answer}
Используем протокол, подробно описанный в задаче (4).


Для $0.99 K$ достаточно проверять не размер множества $S$, а размер множества $(S \times S \times \ldots \times S) = S^l$, которое очевидно лежит в $\mathbf{NP}$, $l$ выбираем так, что $0.99^l \leq \frac{1}{2}$. Получаем сравнение для размеров $M$ и $K$, где $\frac{1}{2} K^l \leq M \leq K^l$.

Ошибку первого рода можно до 0, так как мы получили протокол из $IP$, для которого в одном из эквивалентных определений соответствующая ошибка равна 0.  
\end{answer}


%%%%%%%%%%%%%%%%%%%%%%%%%%%%%%%%%%%%%%%%%%%%%% TASK  5 %%%%%%%%%%%%%%%%%%%%%%%%%%%%%%%%%%%%%%%%%%%%%%
\begin{task} \textbf{5.}
Определим класс $\mathbf{AMA'}$ и $\mathbf{AMA''}$ так: $B \in \mathbf{AMA'} (\mathbf{AMA''})$, если существует полиномиальный алгоритм $V(x,r,s,q)$, такой что:
\begin{itemize}
\item Если $x \in B$, то $\mathbf{Pr}_r[\exists\ s \mathbf{Pr}_q [V(x,r,s,q)=1] \ge \frac{2}{3}] \ge \frac{2}{3}$
\item (для $\mathbf{AMA'}$) Если $x \notin B$, то $\mathbf{Pr}_r[\exists\ s \mathbf{Pr}_q [V(x,r,s,q)=1] \ge \frac{2}{3}] \le \frac{1}{3}$
\item (для $\mathbf{AMA''}$) Если $x \notin B$, то $\mathbf{Pr}_r[\forall\ s \mathbf{Pr}_q [V(x,r,s,q)=1] \le \frac{1}{3}] \ge \frac{2}{3}$
\end{itemize}
(а) (1 балл) Поясните, в чём отличие трёх определений. А именно, почему один и тот же $V$ может удовлетворить одному и не удовлетворить другому. \\
(б) (4 балла) Докажите, что $\mathbf{PP} \subset \mathbf{AMA'}$.\\
(в) (5 баллов) Докажите, что $\mathbf{AMA''}=AMA$.
\end{task}

\begin{answer}

\begin{enumerate}
    \item 
    \item Пусть $\textsf{L} \in \mathbf{PP}$, значит $\exists M$ : 
    \begin{itemize}
        \item $x \in L \implies P_q [M(x,q) = 1] > \frac{1}{2}$
        \item $x \notin L \implies P_q [M(x,q) = 1] \leq \frac{1}{2}$
    \end{itemize}
    Положим тогда $V(x,s,r,q) = M(x,q) \quad \forall r,s$. Тогда :
    \begin{itemize}
        \item $x \in L \implies P_r[ \exists s : P_q[V(x,s,r,q) = M(x,q) = 1] > \frac{1}{2}] = 1 \geq \frac{2}{3}$
        \item $x \notin L \implies P_r[\exists s : P_q[V(x,s,r,q) = M(x,q) = 1] > \frac{2}{3} > \frac{1}{2}] = 0 \leq \frac{1}{3}$
    \end{itemize}
    Заметим, что если в $AMA'$ заменить $\frac{2}{3}$ на $\frac{1}{2}$ , то ничего не изменится, так ка $\frac{1}{2}$ и 
    $\frac{1}{3}$ остались отделимыми. Значит $L \in  AMA'$. 
    \item 
\end{enumerate}
\end{answer}


%%%%%%%%%%%%%%%%%%%%%%%%%%%%%%%%%%%%%%%%%%%%%% TASK  6 %%%%%%%%%%%%%%%%%%%%%%%%%%%%%%%%%%%%%%%%%%%%%%
\begin{task} {\textbf 6.}
Пусть $G$ является генератором псевдослучайных чисел. Рассмотрим следующие модификации:
\begin{itemize}
\item $G'(s) = \left\{
     \begin{array}{lr}
       0^{|G(s)|},&  \text{если } $s$ \text{ содержит ровно } \frac{|s|}{2} \text{ единиц}\\
       G(s) ,& \text{иначе}
     \end{array}
   \right.$
\item $G'(s) = \left\{
     \begin{array}{lr}
       0^{|G(s)|},&  \text{если } $s$ \text{ содержит ровно } \frac{|s|}{3} \text{ единиц}\\
       G(s) ,& \text{иначе}
     \end{array}
   \right.$ 
\end{itemize}
Какие из этих функций являются генераторами псведослучайных чисел и почему?
\end{task}

\begin{answer}
Считаем $n = |s|$. В обоих случая полиномиальная вычислимость $G'(s)$ очевидна. Нужно проверить пункт (2) определения.
\begin{enumerate}
\item \begin{equation} G'(s) = \left\{
     \begin{array}{lr}
       0^{|G(s)|},&  \text{если } $s$ \text{ содержит ровно } \frac{|s|}{2} \text{ единиц}\\
       G(s) ,& \text{иначе}
     \end{array}
   \right. 
   \end{equation}
   $G'(s)$ - не является ГПСЧ. \\
   В $s$ ровно $\frac{|s|}{2}$ единиц в $C_{n}^\frac{n}{2}$ различных $s$. Считая, что $s \sim U_n$ и воспользовавшись тем, что для достаточно больших $n$ выполнено $C_{n}^\frac{n}{2} > \frac{2^n}{n+1}$, получим:
   \begin{equation}
       P(G(s) = 0^{p(n)}) \geq \frac{C_{n}^\frac{n}{2}}{2^n} \geq \frac{1}{n+1}  \quad n \geq N_0
   \end{equation}
   Воспользуемся определeнием вычислительной неотличимости, $y_n \sim U_{p(n)}$, пусть $\{D_n\}$ - такое симейство схем, что $D_n(x) = 1 \iff x = 0^n$. Получим : \\
   $|P\{D_n(G'(s)) = 1\} - P\{D_n(y_n)) = 1\}| \geq \frac{1}{n+1} - \frac{1}{2^n} \geq \frac{1}{2(n+1)}.$ при $n \geq 10$. 
   Также $\frac{1}{2(n+1)} \geq  \frac{1}{2(p(n)+1)}$ при $n > N_{p}$.
   То есть мы получили, что $\exists \{D_n\}$ , $\exists q(p(n)) = \frac{1}{2(p(n)+1)}$  $\forall N \exists n > N : |P\{D_n(G'(s)) = 1\} - P\{D_n(y_n)) = 1\}| \geq \frac{1}{q(p(n))}$. Значит $y_n$ и $G'(s)$ - не являются вычислительно неотличимыми. Значит $G'(s)$ - не является ГПСЧ.

\item $G'(s) = \left\{
     \begin{array}{lr}
       0^{|G(s)|},&  \text{если } $s$ \text{ содержит ровно } \frac{|s|}{3} \text{ единиц}\\
       G(s) ,& \text{иначе}
     \end{array}
   \right.$
$G'(s)$ - не является ГПСЧ. \\ 
В $s$ ровно $\frac{|s|}{3}$ единиц в $C_{n}^\frac{n}{3}$ различных $s$. Воспользуемся формулой Стирлинга: 
\begin{equation}
    C_{n}^{\frac{n}{3}} = \frac{n!}{\frac{n}{3}! \frac{2n}{3}!} \sim \frac{3}{\sqrt{4 \pi n}} \frac{3^n}{2^{\frac{2n}{3}}}
\end{equation}
Обозначим событие $X = \{$ в $s$ ровно $\frac{|s|}{3}$ единиц$\}$. Тогда, считая $s \sim U_n$, получим : 
\begin{equation}
    P\{G'(s) \neq G(s) \} \leq P\{X\} \sim \frac{3}{\sqrt{4 \pi n}} \frac{3^n}{2^{\frac{5n}{3}}}
\end{equation}
$\frac{3^n}{2^{\frac{5n}{3}}} = e^{n(\ln 3 - \frac{5}{3} \ln 2)}$.  Заметим, что $\ln 3 - \frac{5}{3} \ln 2 = c < 0$.
Т.е. $P\{G'(s) \neq G(s) \} \sim \frac{3}{2\sqrt{\pi n}} e^{cn}$. 
Значит $\exists N \forall n > N:$ $P\{G'(s) \neq G(s) \} \leq \frac{3}{\sqrt{\pi n}} e^{c_0 n} \leq e^{cn},\quad c_0, c < 0$. \\

Так как $G(s)$ - ГПСЧ, то $y_n \sim U_{p(n)}, \forall \{D_n\} \forall q_1(x) \text{ - полином } \exists N \forall n > N : $\\ 
$|P\{D_n(G(s)) = 1 \} - P\{D_n(y_n) = 1\}| < \frac{1}{q_1(p(n))}$. 

Воспользуемся определeнием вычислительной неотличимости : \\
$y_n \sim U_{p(n)}, \forall \{D_n\} \forall q(x) \text{ - полином } \exists q_1(x) = \frac{q(x)}{2},  \exists N \forall n > N : 
     |P\{D_n(G'(s)) = 1 \} - P\{D_n(y_n) = 1\}| 
     \leq |P\{D_n(G'(s)) = 1 \} - P\{D_n(G(s)) = 1 \}| + |P\{D_n(G(s)) = 1 \} - P\{D_n(y_n) = 1\}|
     < e^{cn} + \frac{1}{q_1(p(n))} < \frac{1}{q(p(n))}$ 

Получили, что $G'(s)$ и $y_n$ вычислительно неотличимы. Значит $G'(s)$ - ГПСЧ.
\end{enumerate}
\end{answer}




%%%%%%%%%%%%%%%%%%%%%%%%%%%%%%%%%%%%%%%%%%%%%TASK 7%%%%%%%%%%%%%%%%%%%%%%%%%%%%%%%%%%%%%%%%%%%%%$
\begin{task}{\textbf 7.}
Обобщённым судоку называется такая задача: в квадрате $n^2 \times n^2$ в некоторых клетках расставлены числа от $1$ до $n^2$. Вопрос: можно ли заполнить оставшиеся клетки числами от $1$ до $n^2$, так чтобы в каждой строке, в каждом столбце, а также в каждом из $n^2$ "выровненных" квадратов $n \times n$ каждое число встречалось по одному разу. В стандартном судоку $n=3$. Известно, что эта задача $\mathbf{NP}$-полна. Предложите протокол доказательства существования решения с вычислительно нулевым разглашением, не использующий сводимость к какой-либо другой задаче.
\end{task}

\begin{answer} Будем использовать "Протокол привязки к биту", описанный на лекции, для выполнения операций "Загораживание" и "Открытие".
\begin{itemize}
\item $S$ - "загораживание", $S(b) = (c,k)$
\item $R$ - "открытие", $R(c,k) = \{b, \textsf{ERROR}\}$
\item Требования :
    \begin{enumerate}
        \item Корректность : $R(S(b)) = b$
        \item Секретность : привязка к 0 и 1 вычислительно неотличимы.
        \item Неподменяемость : невозможность $R(c,k_0) = 0$ и  $R(c,k_1) = 1$
    \end{enumerate}
\end{itemize} 
\textsf{Протокол:}

Исходная таблица $T_0$. 
\begin{itemize}
    \item $P$ выбирает случайную перестановку $\sigma \in S_{n^2}$, записывает решение в таблицу $T_1$, применяет $\sigma$ к числам $\{1,\ldots n^2\}$ из таблицы $T_1$. "Закрывает" таблицу $T_1$ и перестановку $\sigma$ и отправляет верификатору.
    \item $V$ выбирает случайным образом строку, столбец или квадрат, и просит прувера "открыть" выбранный элемент в $T_1$. Также $V$ выбирает случайную позицию $(i,j)$ в исходной таблице такую, что $T_0[i,j] = x$ (т.е. в $T_0[i,j]$ уже записано некоторое известное число $x$ , и просит прувера "открыть" $\sigma (x)$. Проверяет на корректность строку, столбец или квадрат соответственно, а также проверяет, что $T_1[i,j] = \sigma(x)$.
    \item Повторяем протокол с начала нужное количество раз(полином).
    \item Принимает доказательство, если все проверки пройдены на каждом шаге.
\end{itemize}
Корректность: \\
\begin{itemize}
\item Если решение существует, то $P$ будет дйствовать оптиматьно, ошибка второго рода может возникнуть только в протоколе привязки к биту, но ее можно сделать очень маленькой. Во всех остальных частях протокола ошибки не возникнет.
\item Если решения нет, то либо есть некорректный элемент(строка,столбец, квадрат), либо прувер применил замену индексов не соответствующим перестановке $\sigma$ образом. В первом случае вероятность не заметить ошибку
 $P_{1} \leq \frac{3n^2 - 1}{3n^2} = 1 - \frac{1}{3n^2}$ , во втором, $P_2 \leq \frac{n^2 - 1}{n^2} = 1 - \frac{1}{n^2}$. В любом случае $P \leq 1 - \frac{1}{3n^2}$. $P^{3n^2} \leq (1 - \frac{1}{3n^2})^{3n^2} \sim \frac{1}{e} , n \to \inf $. Значит повторив полиномиальное количество раз сможем получить достаточно малую ошибку.
\end{itemize}
\end{answer}

\begin{task}\textbf{9.}
Расширим определение $\mathbf{PCP}$, введённое на лекции. Назовём классом $\mathbf{PCP}_{c,s}(r,q)_\Sigma$ класс языков $L$, для которых существует полиномиальный вероятностный верификатор $V$ с произвольным доступом к строке $\pi \in \Sigma^*$ длины не более $q2^{O(R)}$, со следующими условиями:
\begin{itemize}
\item $V$ использует не больше $r$ случайных битов и делает не больше $q$ неадаптивных запросов к $\pi$ (обратите внимание, что здесь мы отказываемся от $O(\dot)$-обозначений);
\item Если $x \in L$, то $\mathbf{Pr}\{V^\pi (x) = 1\} \ge c$;
\item Если $x \notin L$, то $\mathbf{Pr}\{V^\pi (x) = 1\} \le s$.
\end{itemize}
Соответственно, класс, введённый на лекции, является классом $\mathbf{PCP}_{1, \frac{1}{2}}(r,q)_{\{0,1\}}$ Будем считать, что размер алфавита $|\Sigma|$ может зависеть от длины входа $|x|$. Докажите, что:\\
(а) (1 балл) Для алфавитов $\Sigma$ полиномиального размера выполнено  $\mathbf{PCP}_{c,s}(O(\log{n}), 0)_\Sigma \subset \mathbf{P}$ \\
(б) (4 балла) Для алфавитов $\Sigma$ полиномиального размера выполнено $\mathbf{PCP}_{c,s}(O(\log{n}), 1)_\Sigma \subset \mathbf{P}$ (для любых параметров $c > s$) \\
(в) (5 баллов) Для алфавитов $\Sigma$ и параметров $q$, таких что $|\Sigma|^q$ не больше полинома, выполнено $\mathbf{PCP}_{1, \frac{1}{|Sigma|q}}(O(\log{n}),q)_\Sigma \subset P$
\end{task}
\begin{answer}
\begin{enumerate}
    \item $q = 0 \implies $ алгоритм проверки детерменированный, к сертификату не обращается, работает полином. Это есть $\mathbf{P}$
\end{enumerate}
\end{answer}

\end{document}
